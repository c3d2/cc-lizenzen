%\documentclass{beamer}
\documentclass[14pt,handout]{beamer}
\usetheme{Darmstadt}
\usepackage{graphicx}
\usepackage{hyperref}
\usepackage[german]{babel}
\usepackage[T1]{fontenc}
\usepackage[utf8]{inputenc}
\setbeamertemplate{footline}[frame number]

\usepackage{pdfcomment}
\newcommand{\ben}[1]{\pdfcomment[author=Ben]{#1}}
\newcommand{\cc}[1]{\includegraphics[height=4mm]{img/#1.png}}
\usepackage{ifthen}
\newcommand{\license}[2][]{\\#2\ifthenelse{\equal{#1}{}}{}{\\\scriptsize\url{#1}}}
\usepackage{textcomp}

\title{Umgang mit Sozialen Netzwerken}
\author{Chaos Computer Club Dresden\\Marius Melzer, Paul Schwanse, Stephan Thamm}
\date{Stand: 22. März 2012}

\begin{document}
\maketitle

\frame{\tableofcontents[hideallsubsections]}

\section{Einleitung}
\subsection{}

\begin{frame}
    \frametitle{Motivation}
    \begin{itemize}
        \item<2-> immer tiefere Integration von Computern in das Leben der Menschen
            \note{Hörgeräte, Herzschrittmacher, Digitale Lebensaufzeichnungen, Bundestrojaner -> FAZ Feuilleton}
        \item<3-> Lücke im Bildungssystem
        \item<4-> sehen uns in der Pflicht Wissen zu vermitteln
        \item<5-> Verständnis anstatt Anleitungen
        \item<6-> Vernetzung mit Schulen und anderen Bildungseinrichtungen
    \end{itemize}
\end{frame}

\begin{frame}
    \frametitle{Wer sind wir?}
    \begin{itemize}
        \item<2-> Chaos Computer Club Dresden (\url{http://c3d2.de})
            \note{}
        \item<3-> Datenspuren (\url{http://datenspuren.de})
        \item<4-> Podcasts (\url{http://pentamedia.de})
        \item<5-> Chaos macht Schule
            \begin{itemize}
                \item \url{http://ccc.de/schule}
                \item \url{http://c3d2.de/schule.html}
            \end{itemize}
            \note{alle Folien auf einmal aufblättern? Ben's vorschlag}
    \end{itemize}
\end{frame}

\begin{frame}
    \frametitle{Chaos macht Schule}
    \begin{itemize}
        \item<2->Ziele:
            \begin{itemize}
                \item<3-> Kinder auf das Internet vorbereiten \ldots
                \item<4-> \ldots nicht das Internet auf Kinder
                    \note{Scheren-Vergleich}
                \item<5-> Informationelle Selbstbestimmung
                \item<6-> Medienkompetenz
                    \note{Medium nicht nur benutzen, sondern auch verstehen. Wir machen keinen Datenschutz-Richtlinien bei Facebook klicken Vortrag!}
                \item<7-> Kreativer Umgang mit Technik
                    \note{Eigene Dinge schaffen, weg von der Konsum-Mentalität}
            \end{itemize}
        \item<8-> Schulklassen
        \item<9-> Elternabende
        \item<10-> Lehrerfortbildung
    \end{itemize}
\end{frame}

\section{Freiheit}
\subsection{}

\begin{frame}
    \frametitle{Was wir vermitteln wollen}
    \begin{itemize}
        \item<2-> Dezentrale Dienste
            \note{man kann sich die Organisation aussuchen die seine Daten bekommt bzw. einen eigenen Server betreiben}
            \begin{itemize}
                \item<3-> Email
                \item<4-> Jabber/XMPP
                \item<5-> Diaspora, Buddycloud
            \end{itemize}
        \item<6-> Alle Sender gleichberechtigt
        \item<7-> Unix-Philosophie von Doug McIlroy:
            \begin{quote}Do one thing, do it right!
            \end{quote}
    \end{itemize}
\end{frame}

\begin{frame}
    \frametitle{Freie Lizenzen}
    \begin{itemize}
        \item<2-> Jeder ist Produzent und Konsument
        \item<3-> Urheberrecht schränkt Verwendung ein
        \item<4-> Freie Lizenzen ermöglichen Verbreitung
%      \ben{an dieser Stelle vielleicht CopyLeft als Buzzword erwähnen (daran erinnern sich dann vielleicht die Leute)}
        \item<5-> Sharing is caring
    \end{itemize}
\end{frame}

\begin{frame}
    \frametitle{Freie Medien}
    \begin{itemize}
        \item<2-> Freie Lehrmaterialien
        \item<3-> Freie Musik
            \begin{itemize}
                \item Jamendo (\url{http://www.jamendo.com/})
                \item Free Music Archive (\url{http://freemusicarchive.org/})
                \item Pentamusic (\url{http://pentamedia.org/})
            \end{itemize}
        \item<4-> Open Clip Art Library (\url{http://openclipart.org/})
        \item<5-> OpenStreetMap (\url{http://openstreetmap.de/})
    \end{itemize}
\end{frame}

\begin{frame}
    \frametitle{Freie Software}
    \begin{itemize}
        \item Linux $ \gets $ Windows
        \item Libre Office/Open Office $ \gets $ Microsoft Office
        \item Firefox $ \gets $ Internet Explorer
        \item Thunderbird $ \gets $ Outlook
        \item Gimp $ \gets $ Photoshop
        \item Inkscape $ \gets $ Illustrator
        \item VLC Media Player $ \gets $ Windows Mediaplayer
    \end{itemize}
\end{frame}


\section{Die 4 Freiheiten}
%herkunft CC Lizenzen
\begin{itemize}
    \item RMS
    \item GPL
    \item 4 Freiheiten:
        \begin[0]{enumerate}
            \item 0
            \item 1 \ldots
        \end{enumerate}
    \item Problem: Software
    \item Was ist mit Urheberrechtlich schützbaren Werken
\end{itemize}
\section{Creative Commons}
\begin{itemize}
    \item Name der Organisation
    \item gegründet 2001
    \item Lawrence Lessig
    \item Entwurf einer Lizenz, die eindeutig und leicht verständlich Nutzung von Werken regelt
    \item 4 Module
\end{itemize}
\section{Die 4 Module der CC Lizenz}
\begin{enumerate}
    \item BY (Namensnennung)
    \item SA (Weitergabe unter gleichen Bedingungen)
    \item ND (keine Abgeleiteten Werke)
    \item NC (nicht kommerzielle Nutzung)
\end{enumerate}
%nachfolgende Logos mit einarbeiten
\section{Modul 0: BY}
\begin{itemize}
    \item 
\end{itemize}<++>
%by kurz erklärt
\section{Modul 1: SA}
%sa kurz erklärt
\section{Modul 2: ND}
%nd kurz erklärt
\section{Modul 3: NC}
%nc kurz erklärt und von abgeraten!!!!!
\section{Die 6 Lizenzen}
\begin{itemize}
    \item durch Kombination 6 Lizenzen zusammenstellbar
        %verweis https://creativecommons.org/choose/?lang=de
    \item Lizensierung des eigenen Werkes einfach möglich
    \item Fallstricke
        % http://wiki.creativecommons.org/Frequently_Asked_Questions#If_I_derive_or_adapt_a_work_offered_under_a_Creative_Commons_license.2C_which_CC_license.28s.29_can_I_apply_to_the_resulting_work.3F
\end{itemize}
\section{Warum CC nutzen?}
\begin{itemize}
    \item \ldots
\end{itemize}<++>
\section{CC Inhalte finden}
\begin{itemize}
    \item 1, 2, 3 links und so
\end{itemize}
\section{Open Educational Resources}
\begin{itemize}
    \item und hier folgt mehr Inhalt
\end{itemize}<++>
\end{document}
