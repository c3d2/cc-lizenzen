\documentclass[table]{beamer}
%\documentclass[14pt,handout]{beamer}
\usetheme{Darmstadt}
\usepackage{graphicx}
\usepackage{hyperref}
\usepackage[german]{babel}
\usepackage[T1]{fontenc}
\usepackage[utf8]{inputenc}
\setbeamertemplate{footline}[frame number]
%\usepackage{enumitem}

\usepackage{pdfcomment}
\newcommand{\ben}[1]{\pdfcomment[author=Ben]{#1}}
\newcommand{\cc}[1]{\includegraphics[height=4mm]{img/#1.png}}
\usepackage{ifthen}
\newcommand{\license}[2][]{\\#2\ifthenelse{\equal{#1}{}}{}{\\\scriptsize\url{#1}}}
\newcommand{\Cc}[1]{\begin{center}
    \includegraphics[height=20mm]{img/200px-Cc-#1.png}
\end{center}}
\usepackage{textcomp}

\usepackage{multirow}
\usepackage{xcolor}


\title{Freie Lizenzen bei Lehr- und Lernmaterialien}
\author{Chaos Computer Club Dresden\\Marius Melzer, Paul Schwanse, Stephan Thamm}
\date{Stand: 13.03.2013}

\begin{document}
\maketitle

\frame{\tableofcontents[hideallsubsections]}

\section{Einleitung}
\subsection{}

\begin{frame}
    \frametitle{Wissen}
    \begin{itemize}
      \item<2-> Einfache Nutzung
      \begin{itemize}
        \item<3-> Verfügbarkeit ohne Zugangsbeschränkungen
        \item<4-> Unabhängig von Zeit und Raum
        \item<5-> Weiterverteilung möglich
      \end{itemize}
      \item<6-> Kollaboratives Erstellen
      \begin{itemize}
        \item<7-> Ermöglicht durch neue Medien
        \item<8-> In gemeinschaftlichem Entwicklungsprozess
        \item<9-> Profitieren von den Änderungen anderer
      \end{itemize}
    \end{itemize}
\end{frame}
 
\begin{frame}
    \frametitle{Wer sind wir?}
    \begin{itemize}
        \item<2-> Chaos Computer Club Dresden (\url{http://c3d2.de})
            \note{}
        \item<3-> Datenspuren (\url{http://datenspuren.de})
        \item<4-> Podcasts (\url{http://pentamedia.de})
        \item<5-> Chaos macht Schule
            \begin{itemize}
                \item<2-> \url{http://ccc.de/schule}
                \item<2-> \url{http://c3d2.de/schule.html}
            \end{itemize}
            \note{alle Folien auf einmal aufblättern? Ben's vorschlag}
        \item<6-> Keine Rechtsanwälte
    \end{itemize}
\end{frame}

\begin{frame}
    \frametitle{Chaos macht Schule}
    \begin{itemize}
        \item<2->Ziele:
            \begin{itemize}
                \item<3-> Kinder auf das Internet vorbereiten \ldots
                \item<4-> \ldots nicht das Internet auf Kinder
                    \note{Scheren-Vergleich}
                \item<5-> Informationelle Selbstbestimmung
                \item<6-> Medienkompetenz
                    \note{Medium nicht nur benutzen, sondern auch verstehen. Wir machen keinen Datenschutz-Richtlinien bei Facebook klicken Vortrag!}
                \item<7-> Kreativer Umgang mit Technik
                    \note{Eigene Dinge schaffen, weg von der Konsum-Mentalität}
            \end{itemize}
        \item<8-> Schulklassen
        \item<9-> Elternabende
        \item<10-> Lehrerfortbildung
        \item<11-> Keine Rechtsberatung
    \end{itemize}
\end{frame}

\begin{frame}
    \frametitle{Urheberrechtsverletzung?}
    \begin{itemize}
        \item<2-> Hintergrund-Musik in einem eigenen Video
        \item<3-> Bild aus Karrikatur/Studie
        \item<4-> Austeilen einer CD mit Lied
        \item<5-> Zeigen von Filmen im Unterricht
        \item<6-> Text aus einem Arbeitsbuch kopieren
        \item<7-> Verwenden eines Artikels aus dem Internet
    \end{itemize}
\end{frame}

\section{Urheberrecht}
\subsection{}

\begin{frame}
  \frametitle{Urheberrecht (1)}
    \begin{itemize}
        \item<2-> Urheberrecht worauf?
        \begin{itemize}
          \item<3-> Sprachwerke \dots
          \item<4-> Werke der Musik
          \item<5-> pantomimische Werke \dots
          \item<6-> \dots
          \item<7-> persönliche geistige Schöpfungen
        \end{itemize}
      \item<8-> Urheber hat automatisch Urheberrecht
      \item<9-> Urheberpersönlichkeitsrechte
      \item<10-> Verfielfältigung, Verbreitung, \dots
    \end{itemize}
\end{frame}

\begin{frame}
  \frametitle{Urheberrecht (2)}
    \begin{itemize}
        \item<2-> erlischt 70 Jahre nach Tod
        \item<3-> Schranken
        \begin{itemize}
          \item<4-> Privatkopie
          \item<5-> Zitate
          \item<6-> Unterricht und Forschung
        \end{itemize}
        \item<7-> Verwertungsgesellschaften
        \item<8-> Lizenzen
    \end{itemize}
\end{frame}

\section{Freie Lizenzen}
\subsection{}

\begin{frame}
    \frametitle{Die 4 Freiheiten}
    \begin{itemize}
        \item<2-> Die Software darf \dots
            \begin{enumerate}
            \setcounter{enumi}{-1}
            %startwert -1, das nächste Item ist also "`0"'
                \item<3-> zu jedem Zweck ausgeführt werden
                \item<4-> untersucht und verändert werden
                \item<5-> verbreitet werden
                \item<6-> verbessert, und die Verbesserung verbreitet werden
            \end{enumerate}
        \item<7-> Auf Software zugeschnitten
    \end{itemize}
\end{frame}

\begin{frame}
    \frametitle{Creative Commons}
    \begin{itemize}
        \item<2-> Organisation hat sich zum Ziel gesetzt, Lizenzen zu erarbeiten
        \item<3-> treibende Kraft: Lawrence Lessig
        \item<4-> 2001 in den USA gegründet
        \item<5-> 2012: 10-jähriges bestehen CC Europe
        \item<6-> Eine Lizenz, die ...
            \begin{itemize}
                \item<7-> eindeutig ist
                \item<8-> leicht verständlich ist
                \item<9-> Nutzung von Werken regelt
            \end{itemize}
        \item<10-> Version 4 wird aktuell erarbeitet
        \item<11-> Besteht aus 4 Modulen: BY, SA, ND, NC
    \end{itemize}
\end{frame}

%nachfolgende Logos mit einarbeiten
\begin{frame}
    \frametitle{Modul 0: BY (By)}
    \begin{itemize}
        \item Namensnennung des Original Urhebers
        \item impliziert ab CC 3.0
        \item Urheberverweis kann auf Wunsch zurückgezogen werden
    \end{itemize}
        \Cc{by}
\end{frame}

%by kurz erklärt
\begin{frame}
    \frametitle{Modul 1: SA (ShareAlike)}
    \begin{itemize}
        \item Weitergabe unter gleichen Bedingungen (s. Kompatibilitätstabelle)
        \item sichert, dass gleiche Lizenz erhalten bleibt
    \end{itemize}
    \Cc{sa}
\end{frame}

%sa kurz erklärt
\begin{frame}
    \frametitle{Modul 2: ND (NonDerivative)}
    \begin{itemize}
        \item Keine Bearbeitung
        \item Werk darf frei kopiert und weitergegeben werden
        \item Werk darf nicht bearbeitet werden, d. h. Verwendung in einer Collage nicht möglich
    \end{itemize}
    \Cc{nd}
\end{frame}

%nd kurz erklärt
\begin{frame}
    \frametitle{Modul 3: NC (NonCommercial)}
    \begin{itemize}
        \item Nur nicht-kommerzielle Nutzung
        \item Probleme: Was ist kommerzielle Nutzung?
    \end{itemize}
    \Cc{nc}
\end{frame}

%nc kurz erklärt und von abgeraten!!!!!

\begin{frame}
    \frametitle{Baukastensystem}
    \begin{itemize}
        \item<2-> durch Kombination 6 Lizenzen zusammenstellbar
            \url{https://creativecommons.org/choose/?lang=de}
        \item<3-> Lizensierung des eigenen Werkes einfach möglich
        \item<4-> Es muss vermerkt werden:
            \begin{enumerate}
                \item<5-> Original Urheber
                \item<6-> Lizenz
                \item<7-> Link zur Lizenz
            \end{enumerate}
        \item<8-> nicht exklusiv
        \item<9-> Nutzungsbedingungen der Inhalte ist Nutzern klar, da diese die Lizenz erkennen
        \item<10-> kann nicht zurückgezogen werden
            % hieraus ergeben sich die Fragen: Was passiert, wenn Werk unter mir nicht genehmen Umständen veröffentlicht wird?
            % kann ich mein Werk anders lizensieren?
    \end{itemize}
\end{frame}

\begin{frame}
    \frametitle{Kompatibilität zwischen Lizenzen}
    \begin{center}
        \begin{tabular}{ c l|c|c|c|c|c|c|c }
            \multicolumn{2}{c|}{}&  \multicolumn{7}{ c }{Lizenzen neues Werk} \\
            \cline{3-9}
            \multicolumn{2}{c|}{Org. Lizenz}& by & by-nc & by-nc-nd & by-nc-sa & by-nd & by-sa & pd \\
            \hline
            \multirow{7}{*}{} & pd & \cellcolor{green} & \cellcolor{green} & \cellcolor{green} & \cellcolor{green} & \cellcolor{green} & \cellcolor{green} & \cellcolor{green} \\
            & by & \cellcolor{green} & \cellcolor{green} & \cellcolor{green} & \cellcolor{green} & \cellcolor{green} & \cellcolor{green} &  \\
            & by-nc &  & \cellcolor{green} & \cellcolor{green} & \cellcolor{green} &  &  &  \\
            & by-nc-nd &  &  &  &  &  &  &  \\
            & by-nc-sa &  &  &  & \cellcolor{green} &  &  &  \\
            & by-nd &  &  &  &  &  &  &  \\
            & by-sa &  &  &  &  &  & \cellcolor{green} &  \\
        \end{tabular}
    \end{center}
\end{frame}

\begin{frame}
    \frametitle{Weitere freie Lizenzen}
    \begin{itemize}
        \item<2-> Open Database License (ODbL)
        \item<3-> Charityware
        \item<4-> Pizzaware
        \item<5-> eigene Lizenztexte und Derivate
        \item<6-> Unbekannte Lizenzen sind Mehraufwand für Nutzer
    \end{itemize}
\end{frame}

\section{Projekte}
\subsection{}

\begin{frame}
    \frametitle{Open Educational Resources}
    \begin{itemize}
        \item<2-> Materialien sind wie folgt zugänglich:
            \begin{itemize}
                \item<3-> ohne Einschränkung
                \item<4-> in einem freien Format
                \item<5-> unter einer offenen Lizenzen 
            \end{itemize}
        \item<6-> Begriff umfasst alle Arten von (Lehr-)Materialien
        \item<7-> Idee: Wissensunterschiede zwischen Industrienationen und Entwicklungsländern abbauen
        \item<8-> 2007: Cape Town Open Education Declaration
    \end{itemize}
\end{frame}

\begin{frame}
    \frametitle{Open Courseware}
      \begin{itemize}
        \item<2-> timm
        \item<3-> Berkley
        \item<4-> MIT
        \item<5-> Coursera
        \item<6-> Edutags
    \end{itemize}
\end{frame}

\begin{frame}
    \frametitle{Wikimedia}
    \begin{itemize}
      \item<2-> Wikipedia
      \item<3-> Wiki Commons
      \item<4-> Wikibooks
      \item<5-> Wikiversity
      \item<6-> Wictionary
      \item<7-> Wikiquotes
      \item<8-> ... und 12 weitere
    \end{itemize}
\end{frame}

\begin{freme}
    \frametitle{E-Learning}
    \begin{itemize}
        \item<2-> Opal
        \item<3-> Sächsischer Bildungsserver
        \item<4-> \ldots
    \end{itemize}
\end{frame}

\begin{frame}
    \frametitle{CC Inhalte finden}
    \begin{itemize}
        \item<2-> Creative Commons Search
        \item<3-> Creative Commons Content Directories
        \item<4-> Google Web und Images
        \item<5-> Flickr
        \item<6-> 500px
        \item<7-> Jamendo
        \item<8-> ccMixter
        \item<9-> Bundesarchiv
        \item<10-> \ldots
    \end{itemize}
\end{frame}

\begin{frame}
    \frametitle{Und weiter?}
    \begin{itemize}
      \item<2-> Freie Materialien \dots
      \item<3-> \dots~in offenen Formaten \dots
      \item<4-> \dots~erstellt mit freien Werkzeugen \dots
      \item<5-> \dots~auf freien (Betriebs-)Systemen \dots
      \item<6-> \dots~und offener Hardware.
    \end{itemize}
\end{frame}

\section{Fazit}
\subsection{}

\begin{frame}
    \frametitle{Fazit}
    \begin{itemize}
      \item Rechtslage kompliziert
      \item Lizenzen zum einfachen ``Befreien'' von Inhalten
      \item Wissen baut auf Wissen auf
      \item Kollaboration ist wichtig, erfordert aber Abgabe von Rechten
      \item (Wissens-)Austausch muss sich auch in Lizenzen wiederspiegeln
      \item Eine Gesellschaft von Prosumenten muss eine Tauschkultur leben, um die Kreativität eines einzelnen nicht zu unterdrücken
    \end{itemize}
\end{frame}

\begin{frame}
    \frametitle{Diskussion}
    \begin{itemize}
        \item Vielen Dank für ihre Aufmerksamkeit
        \item \url{http://c3d2.de/schule.html}
        \item \url{schule@c3d2.de}
        \item Für weitere Informationen (u.a. diese Folien) besuchen Sie bitte \url{http://c3d2.de/schule.html}
    \end{itemize}
    \begin{center}
   Folien vom Chaos Computer Club Dresden\\
   {\cc{by-sa}}
   \end{center}
\end{frame}

\end{document}
